% Metódy inžinierskej práce

\documentclass[12pt,twoside,slovak,a4paper]{article}

\usepackage[slovak]{babel}
%\usepackage[T1]{fontenc}
\usepackage[IL2]{fontenc} % lepšia sadzba písmena Ľ než v T1
\usepackage[utf8]{inputenc}
\usepackage{graphicx}
\usepackage{url} % príkaz \url na formátovanie URL
\usepackage{hyperref} % odkazy v texte budú aktívne (pri niektorých triedach dokumentov spôsobuje posun textu)
\usepackage{geometry}
\geometry{
  a4paper,
  top=25mm, 
  right=15mm, 
  bottom=25mm, 
  left=15mm
}
\usepackage{cite}
%\usepackage{times}

\pagestyle{headings}

\title{Modelovanie webových stránok.\\Vytvaranie Frontend.\thanks{Semestrálny projekt v predmete Metódy inžinierskej práce, ak. rok 2021/22, vedenie: Ing. Fedor Lehocki, PhD.}} % meno a priezvisko vyučujúceho na cvičeniach

\author{Yehor Lykhachov\\[2pt]
	{\small Slovenská technická univerzita v Bratislave}\\
	{\small Fakulta informatiky a informačných technológií}\\
	{\small \texttt{xlykhachov@is.stuba.sk}}
	}

\date{\small 6. november 2021} % upravte



\begin{document}

\maketitle





\section{Úvod}

V tomto článku som opísal základné princípy fungovania webových stránok z klientskej strany. Stanovil som hlavný cieľ svojho článku: popis všetkých procesov
vytvárania, ako aj pracovných procesov front-end. Chcem ukázať, aká
dôležitá je táto oblasť vývoja, rozvoj na strane klienta pri
vytváraní webových stránok. Samozrejme, zdôraznil som časť svojho článku, ktorá
popisuje znalosti a schopnosti, ktoré musí front-end vývojár mať.\\
\\
Článok je rozdelený do dvoch častí(part 1 - 7/10, part 2 - 3/10), z
ktorých jedna berie do úvahy technickú zložku a druhá popísuje potrebné
nástroje, znalosti a zručnosti, ktoré by dobrý front-end špecialista mal mať.
Technickými aspektmi sa podrobne zaoberá Part~\ref{Part 2}.
Znalosti a techniky, ktoré potrebuje špecialista na frontend, sú zahrnuté v Part~\ref{Part 3}. 
Záverečné poznámky prináša časť~\ref{zaver}.



\section{Technická časť} \label{Part 2}
\subsection{Z čoho sa skladá website?}
Web-site je súbor verejne prístupných, vzájomne prepojených webových stránok, ktoré používajú jeden názov domény. Webové stránky môže vytvárať a spravovať jednotlivec, skupina, podnik alebo organizácia na rôzne účely. Všetky verejne prístupné webové stránky spolu tvoria World Wide Web.\label{sub2.1}\cite{zdroj_2}\\
\\
Zjednodušene povedané, webová stránka je adresa umiestnená na internete, ktorá obsahuje určité informácie (text, video, fotografie, dokumenty, hudbu atď.).\\
\\
Web-site pozostáva z:\\
\\
- Názov domény (doména) - adresa.\\
- Server (hosting) - počítače/servery pripojené k World Wide Webu, na ktorých sú umiestnené súbory s webovými zdrojmi.\\
- Súbory - zvyčajne CMS (systém správy obsahu), pomocou ktorého je možné pohodlne vytvárať a spravovať webové stránky. Alebo to môže byť len jedna alebo niekoľko statických html stránok (s obrázkami, css, js), ktoré je potrebné upravovať ručne.\\
- Obsah webových stránok: obrázky, videá, text a iné súbory.\\
\subsection{Čo je frontend?} \label{sub2.2}
Hlavná a možno najdôležitejšia otázka tohto článku. Mnohí ľudia ešte stále úplne nerozumejú tomu, o čom je povolanie frontendového vývojára. Čo presne tento človek robí, čo musí vedieť, aká je jeho úloha a mnoho ďalších otázok, na ktoré veľa ľudí nepozná odpoveď.
V tomto článku sa pokúsim vysvetliť, aké znalosti a technické zručnosti sú potrebné pre človeka, ktorý sa rozhodol stať frontendovým vývojárom.\\
\\
Zjednodušene povedané, frontend je čokoľvek, čo môže prehliadač čítať, zobrazovať a/alebo spúšťať.  To znamená, že ide o HTML, CSS a JavaScript.\\
\\
Jazyk HTML (HyperText Markup Language) hovorí prehliadaču, čo je obsahom stránky, napr. header, table, article, image.\\
\\
CSS (Cascading Style Sheets) hovorí prehliadaču, ako má zobrazovať prvky na stránke, napr. "druhý odsek je odsadený o 10 pixelov" alebo "text v tomto bloku by mal byť biely a písaný písmom Impact".\\
\\
JavaScript pomocou odľahčeného programovacieho jazyka určuje prehliadaču, ako má reagovať na určité interakcie.Väčšina stránok v skutočnosti nepoužíva veľa JavaScriptu, ale ak napríklad na niečo kliknete a zmení sa to bez opätovného načítania stránky alebo prechodu na inú stránku, znamená to, že bol niekde použitý JavaScript.\\
\\
To všetko je úzko prepojené s rôznymi druhmi rámcov, knižníc a množstvom ďalších rovnako dôležitých vecí pri vývoji webových stránok. V nasledujúcom článku sa venujeme každej z uvedených hlavných zložiek samostatne.

\subsection{HTML} \label{sub2.3}
HTML stands for Hypertext Markup Language which is the most fundamental building
block of the World Wide Web (WWW). It was originally designed to be used as a markup
language for scientific documents, but its overall design has enabled it to be adopted by
other fields over the years (WHATWG, 2020).\cite{zdroj_1}\\
\\
Jazyk HTML používa na zobrazenie textu, obrázkov a iného obsahu vo webovom prehliadači značky. Značky HTML obsahujú špeciálne "elementy", ako napríklad <head>, <title>, <body>, <header>, <footer>, <article>, <section>, <p>, <div>, <span>, <img>, <aside>, <audio>, <nav>, <video> a mnoho ďalších.\\
\\
Element HTML je oddelený od zvyšku textu v dokumente pomocou "značiek", ktoré pozostávajú z názvu elementu obklopeného znakmi < a >. Názov prvku v rámci značky nerozlišuje veľkosť písmen. To znamená, že sa môže písať veľkými alebo malými písmenami, prípadne zmiešane. Napríklad značka <title> môže byť zapísaná ako <Title>, <TITLE> alebo iným spôsobom.\\
\\
Although the syntax of HTML provides an intuitive way to inspect the structure and
hierarchy of a web page’s elements, the way it is designed to be used dictates that every
single HTML file is served as an individual web page (WHATWG, 2020). For example,
the display language of a website in most cases is consistent throughout all pages, but
since one HTML file represents one page, the value of `lang` property inside `<html>`
tags will be repeated in multiple HTML files.\cite{zdroj_1}
\\
\subsection{CSS} \label{sub2.4}
Cascading Style Sheets (CSS) was first introduced in 1996. At that moment, the current
version of HTML was HTML2. HTML is responsible for the layout, the appearance, and
the information input of the page. CSS is responsible for describing the presentation of
a document written in a markup language like HTML. It is a fundamental technology of
the World Wide Web, alongside HTML and JavaScript (CSS WG, 2020).\cite{zdroj_1}\\
CSS sa používa na definovanie štýlov (pravidiel) pre dokumenty - vrátane dizajnu a variantov rozloženia pre rôzne zariadenia a veľkosti obrazovky. Tento spôsob formátovania má niekoľko výhod:\\
\\
--značky nie sú duplikované;\\
--dokument sa ľahšie udržiava;\\
--vzhľad celého webu sa dá meniť centrálne, namiesto toho, aby sa upravovalo formátovanie jednotlivých stránok.\\
\\
CSS was created to provide a tool for web designers to manage the style of HTML
elements. Moreover, it was not intended to manage the visual look of webpages; it was
only designated to access and control the properties of particular elements that existed
in the HTML file. As the time CSS was created, most of the websites looked plain and
architects could not anticipate the richness and complexity of the designs of the current
state of webpages nowadays. The reason for that was clearly stated by one of the
creators of CSS:\\
“Variables, constants, conditionals, expressions over variables, etc. are features that
used most by programmers, however, they do not exist in CSS. These things give a lot
of power to developers as they know how to customize and utilize these features to their
best. However, for inexperienced users, they will feel more pressure, more likely, feel
scared and stop using CSS” (Bos, n.d.).\cite{zdroj_1}
\subsection{JavaScript} \label{sub2.5}
JavaScript je programovací jazyk, ktorý umožňuje vytvárať dynamicky aktualizovaný obsah, spravovať multimédiá, animovať obrázky a vlastne robiť čokoľvek iné. Dobre, nie všetko, ale aj tak je úžasné, čo všetko môžete dosiahnuť pomocou niekoľkých riadkov kódu JavaScript.\cite{zdroj_3}\\
\\
Jadro jazyka JavaScript pozostáva z niekoľkých spoločných funkcií, ktoré umožňujú:\\
\\
- Ukladanie údajov do premenných. \\
- Operácie s textovými fragmentmi.\\
- Spustite kód podľa konkrétnych udalostí, ktoré sa vyskytnú na webovej stránke.\\
- A ešte oveľa viac!\\
\\
Kód JavaScriptu sa spustí pomocou mechanizmu JavaScriptu prehliadača po spracovaní kódu HTML a CSS a vytvorení webovej stránky. Tým sa zabezpečí, že štruktúra a štýl stránky sú v čase spustenia JavaScriptu už vytvorené.\\
\\
To je dobré, pretože JavaScript sa často používa na dynamickú zmenu HTML a CSS s cieľom aktualizovať používateľské rozhranie prostredníctvom rozhrania API objektového modelu dokumentu. Ak by sa JavaScript spustil pred načítaním HTML a CSS, viedlo by to k chybám.  
\subsection{Frameworky} \label{sub2.6}
V oblasti frontendov existuje mnoho frameworkov a knižníc, ktoré musí používať každý frontendový vývojár. V grafe sú znázornené najobľúbenejšie frameworky posledných rokov.\cite{zdroj_2}\\
\\
\includegraphics[width = 400pt]{diagram.pdf}\\
\\
\begin{tabular}{|c|c|}
    Top 7 frameworks & in 2021 \\
    React & 74.2\% \\
    Angular & 33.4\% \\
    Vue.js & 29.9\% \\
    Svelte & 11.8\% \\
    Ember.js & 6.5\% \\
    Preact & 5.6\% \\
    Blackbone.js & 4\% \\
    
\end{tabular}

\section{Časť o frontendovom vývojárovi} \label{Part 3}
Frontendový vývojár nevytvára len rozvrhnutie. Má dobré znalosti JavaScriptu, rozumie frameworkom a knižniciam (a niektoré z nich aktívne používa), rozumie tomu, čo je "pod kapotou" na strane servera. Nebojí sa preprocesorov a builderov LESS, SASS, GRUNT, GULP, vie pracovať s DOM, API, SVG-objektmi, AJAXom a CORS, vie robiť SQL dotazy a hrabať sa v dátach. Výsledkom je zmes zručností, ktoré sú doplnené o pochopenie princípov dizajnu UI/UX, adaptívneho a responzívneho rozvrhnutia, zručností v oblasti vývoja naprieč prehliadačmi a platformami a niekedy aj mobilných zariadení.\\
\\
Front-end developer musí tiež vedieť pracovať s riadením verzií (Git, GitHub, CVS atď.), používať grafické editory a "hrať sa" so šablónami z rôznych CMS.\\
\\
Taktiež je veľmi žiaduce vedieť anglicky, aby ste nemuseli špecifikáciu prekladať v Google Translator, vedieť pracovať v tíme, niekedy viacjazyčne, rozumieť webovým fontom a rozumieť testerom a procesu testovania.\\
\\
Ktoré technológie by mal vývojár front-endu poznať:
\\
- HTML a CSS
\\
- Preprocesory CSS (Sass, Less, Stylus atď.)
\\
- JavaScript
\\
- Populárne frameworky a knižnice: jQuery,Angular.JS,React.JS, Backbone.js atď.
\\
- OOCSS / BEM / SMACSS
\\
- SVG
\\
- DOM
\\
- ROZHRANIE HTML5 API
\\
- ECMAScript 6
\\
- Populárne CMS (WordPress, Drupal, Joomla atď.)
\\
- Rozumieť princípom tvorby backendu a rozumieť serverovým technológiám (Node.js, PHP, Ruby, .NET atď.)
\\
- Nástroje na ladenie (Chrome Dev Tools, Firebug atď.)
\\
- Transpilátory JavaScript (Babel)
\\
- Nástroje na správu verzií (Git, GitHub, CVS atď.)
\\
- Databázy a dotazovacie jazyky (SQL, MySql, NoSQL, MongoDB atď.)
\\
- Grafické editory (Photoshop, Illustrator atď.)

\paragraph{Historické súvislosti.}
História informatiky je pomerne široká téma a veľkú časť tejto histórie zaberá aj frontend. Nie je však možné povedať všetko v skratke. Najprv tu bola tvrdá konkurencia prehliadačov s chaotickými inováciami, potom pokusy o štandardizáciu, ktoré komunita odmietla (XHTML) alebo zlyhali kvôli prílišným ambíciám (ECMAScript 4), potom stagnácia a potom produktívna práca v spolupráci s komunitou. HTML a JavaScript sa stali živými štandardmi, ktoré sú každoročne aktualizované a podporované všetkými prehliadačmi.
\paragraph{Technológia a ľudia.}
Frontend je v súčasnosti jednoznačne jednou z najobľúbenejších a najvyhľadávanejších oblastí IT. Koniec koncov, internet dnes používa každý, teda aj webové stránky. Webová stránka nemôže existovať bez frontendu. Niektorí ľudia si možno neuvedomujú, aký dôležitý je dnes frontend. V každom prípade má však frontend obrovský vplyv na životy ľudí.

\section{Záver} \label{zaver}
V tomto článku som opísal základné komponenty a princípy vytvárania frontend, ktoré potrebuje každý frontendový vývojár. V tomto článku som sa venoval tomu, z čoho sa skladá webová stránka, čo sú HTML, CSS a JavaScript a aké sú najpopulárnejšie frameworky. A samozrejme som odpovedal na hlavnú otázku článku - čo je to frontend.
\bibliography{literatura}
\bibliographystyle{plain}
\end{document}
